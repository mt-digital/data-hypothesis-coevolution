% !TEX TS-program = xelatex
% !TEX encoding = UTF-8 Unicode
\documentclass[11pt]{amsart}
\usepackage{fullpage}
\usepackage{setspace}

\usepackage[foot]{amsaddr}
\usepackage{graphicx}
\usepackage[usenames,dvipsnames]{xcolor}
\usepackage[]{geometry}
%\usepackage{mathtools}
% \usepackage{subfigure}
\usepackage{subcaption}
\usepackage{booktabs}
\usepackage{lineno}
\usepackage{hyperref}
\usepackage{float}
\usepackage{natbib} %this allows for styles in referencing
%\bibpunct[, ]{(}{)}{,}{a}{}{,}
\DeclareMathOperator{\var}{var}
\DeclareMathOperator{\cov}{cov}
\DeclareMathOperator{\E}{E}

\synctex=1

\newcommand*\patchAmsMathEnvironmentForLineno[1]{%
  \expandafter\let\csname old#1\expandafter\endcsname\csname #1\endcsname
  \expandafter\let\csname oldend#1\expandafter\endcsname\csname end#1\endcsname
  \renewenvironment{#1}%
     {\linenomath\csname old#1\endcsname}%
     {\csname oldend#1\endcsname\endlinenomath}}%
\newcommand*\patchBothAmsMathEnvironmentsForLineno[1]{%
  \patchAmsMathEnvironmentForLineno{#1}%
  \patchAmsMathEnvironmentForLineno{#1*}}%
\AtBeginDocument{%
\patchBothAmsMathEnvironmentsForLineno{equation}%
\patchBothAmsMathEnvironmentsForLineno{align}%
\patchBothAmsMathEnvironmentsForLineno{flalign}%
\patchBothAmsMathEnvironmentsForLineno{alignat}%
\patchBothAmsMathEnvironmentsForLineno{gather}%
\patchBothAmsMathEnvironmentsForLineno{multline}%
}

\usepackage{float}
%\usepackage{lmodern}
%\usepackage{unicode-math}
\usepackage{mathspec}
\usepackage{xltxtra}
\usepackage{xunicode}
\defaultfontfeatures{Mapping=tex-text}
%\setsansfont[Scale=MatchLowercase,Mapping=tex-text]{Helvetica}
%\setmonofont[Scale=0.85]{Bitstream Vera Sans Mono}
%\setmainfont[Scale=1,Ligatures={Common}]{Minion Pro}
%\setromanfont[Scale=1,Ligatures={Common}]{Minion Pro}
%\setmathrm[Scale=1]{Minion Pro}
%\setmathfont(Digits,Latin)[Numbers={Lining,Proportional}]{Minion Pro}

\definecolor{linenocolor}{gray}{0.6}
\renewcommand\thelinenumber{\color{linenocolor}\arabic{linenumber}}

\usepackage{fix-cm}

\usepackage{tikz}
\tikzset{every state/.style={minimum size=.5in}}
\usetikzlibrary{arrows}
\usetikzlibrary{positioning,calc}

\usetikzlibrary{er}
\usetikzlibrary{shapes,snakes}
\usetikzlibrary{automata}

\setlength{\footskip}{70pt} %set page number lower down on the page


%\usepackage{hanging}

\setcounter{totalnumber}{1}

\newcommand{\mr}{\mathrm}
\newcommand{\tsc}[1]{\text{\textsc{#1}}}
\calclayout %centers text on page.
\begin{document}


\title{}
\author[Turner]{Matthew A.~Turner}

%Author order subject to change
\address{$^1$Cognitive and Information Sciences, University of California, Merced, Merced, CA, USA}
\email{maturner01@gmail.com}

\date{\today}

\maketitle

% \tableofcontents

{\vspace{-6pt}\footnotesize\begin{center}\today\end{center}\vspace{12pt}}

\linenumbers
\modulolinenumbers[3]


\begin{abstract}
\noindent \small
  What factors contribute to the rate of population-level innovation? Why do
  some cultures seem to innovate faster than others? Is it driven by selection,
  or are there information-theoretic reasons related to the nature of 
  replicating agents? I explore this question here where I propose a model
  of innovation as it depends on the likelihood landscape of cultural models
  on the observation of cultural communication, which could also be called
  cultural artifacts, interpreted loosely as stuff people (or agents more generally)
  said. 
\end{abstract}

\doublespacing

\vspace{12pt}

\noindent \textbf{Keywords:} population dynamics, information theory, 
bayesian learning, culture, innovation, complexity


%%%%%%%%%%%%%%%%%%%%%%%%%%%%%%%%%%%%
%%  	Introduction
%%%%%%%%%%%%%%%%%%%%%%%%%%%%%%%%%%%%
\section{Introduction}

WEIRD culture may not be universal, but it seems to be exploding in complexity.
More complexity seems to breed more innovation, since greater recombination
is possible. Take modern electric cars, for example. They are built of many
earlier innovations, such as innovations in body and frame materials, battery
innovation, and aerodynamic innovation---each of these innovations were in turn
based on past innovations. While this seems qualitatively or anecdotally true,
in order to pursue the subject scientifically, we must have quantitative 
theories and models of the relationship between innovation and cultural complexity.
In this paper I suggest a dynamical evolutionary
model used in computational experiments that will illuminate the relationship
between cultural complexity and innovation. The model is an implementation of
the coevolution model of \cite{Ferdinand2018} where both the rates of public 
productions of cultural variants and private mental models---``hypotheses'' about
how culture is---co-evolve guided by the ``fitness'' of public productions
given the rate of private cultural models. 

In this model, learners are conceptualized as identical bayesian learners.
Bayesian learning has found application in models of language learning in
the form of iterated learning models. Kalish and Griffiths showed that iterated
learning of cultural artifacts or cultural models correspond to the information
process of maximum a posteriori (MAP) sampling and expectation maximization
sampling. Ferdinand (2018) demonstrated the usefulness of a model
of the coevolution of data and hypotheses. Evolutionary models of data, i.e.
cultural ``artifacts'' or ``variants,'' often take the form of the replicator
equation; the fitness of each variant can be interpreted as the net benefit
of producing that cultural variant given the distribution of hypotheses, or
possible cultural models, in the population. But the proportion of agents using
each cultural model also evolves. The evolution of cultural models is the
subject of iterated learning models (ILM). 


\subsection{Model}

\cite{Ferdinand2018} uses these two formulations to derive two recurrence equations for the 
coevolution of variants and models:

\begin{subequations}
  \begin{equation}
    p(d_i)' = \sum_j \frac{p(h_j)p(d_i|h_j)}{\sum_{k=1}^K p(d_k)p(d_k|h_j)} p(d_i) \\
  \end{equation}
  \begin{equation}
    p(h_j)' = \sum_i \frac{p(d_i)p(d_i|h_j)}{\sum_{k=1}^K p(h_k)p(d_i|h_k)} p(h_j).
  \end{equation}
\end{subequations}
\noindent

We can write these in matrix-vector form as 

\begin{subequations}
  \begin{equation}
    p(d)' = \left(D^{} p(h)\right) \odot p(d)
  \end{equation}
  \begin{equation}
    p(h)' = \left(H^T p(d)\right) \odot p(h),
  \end{equation}
\end{subequations}
\noindent

where $\odot$ represents element-wise multiplication.

$D$ is the data transition matrix and $H^T$ is the hypotheses transition matrix.
These rely on the ``fitness'' matrix of likelihoods, $W_{ij} = p(d_i|h_j)$ and
the global fitness of hypothesis $h_j$, $Z_{d,j} = \sum_k p(d_k)p(d_k|h_j)$,
and the global fitness of artifact $d_i$, $Z_{i,h} = \sum_k p(h_k)p(d_i|h_k)$:

\[
  D = \begin{pmatrix} W_{i=1}/Z_{d,1}~|~W_{i=2}/Z_{d,2}~|~\cdots~|~W_{i=K}/Z_{d,K} \end{pmatrix}
\]

\[
  H = \begin{pmatrix} W_{j=1}/Z_{1,h} \\
                      W_{j=2}/Z_{2,h} \\
                       \vdots         \\ 
                      W_{j=K}/Z_{K,h} \end{pmatrix}
\]

where $W_i$ are the columns and $W_j$ are the rows of $W$. Each element of 
either column $i$ or row $j$ is divided by the divisor $Z_x$ above.
In this report, $W$ is constant throughout the simulation, but $Z_d$ and $Z_h$
are clearly not, since $p(d)$ and $p(h)$ evolve.

\subsection{Outcome Measures}

Recall the goal here is to understand the relationship between innovation rate in
a culture and the cultural complexity of that culture. There are two ways
I propose to mesure innovation rate. The first is indirect, and may not even
measure ``innovation'' per se, but I think it belongs: that is to calculate the
time required to converge to a solution as a function of cultural complexity, 
which I denote $T(K)$. Since iterated learning can be thought of as an expectation
maximization algorithm, this time to converge represents the computational 
complexity for a given a cultural complexity 
\citep{Griffiths2007a,Kalish2007}. We can also investigate the relationship
between the innovation rate at time $t$, which for probability distributions can be
measured with the Kullback-Liebler divergence from one timestep to another, 

\begin{equation}
  I_x(t) = KL(p^{(t+1)}(x)~||~p^{(t)}(x))
\end{equation}

where $x \in \{d, h\}$. See \cite{Griffiths2013} for a similar approach that also
provides experimental evidence of information rate as a function of cultural
parameters.

\subsection{Experiments}

I ran N experimental trials for each value of $K$ tested. From this we can 
count the average number of timesteps required for convergence for each $K$,
and create a timeseries of innovation for each $K$. 


\section{Results}


\begin{figure}[H]
  \caption{Time to converge as a function of cultural complexity. As culture
  becomes more complex, the ``computation'' social interaction implements 
  takes longer and longer to converge, following a quadratic relationship,
  meaning the cultural computation is $O(K^2)$.}
  \centering
    \includegraphics[width=0.8\textwidth]{Figures/AveT_v_K.pdf}
  \label{fig:}
\end{figure}

% \begin{figure}[H]
%   \caption{Innovation rate over time for different cultural complexities.}
%   \centering
%     % \includegraphics[]{Figures/}
%   \label{fig:}
% \end{figure}



\section{Discussion}

The model presented here makes many implicit assumptions that are worth 
explaining in detail because they present opportunities for future extensions
that will help us further understand cultural dynamics. First, the model assumes
that agents interact with equal probability, and that there are many agents.
Second, there is perfect communication in this model. 
Third, an interpretive assumption is that these cultural data and hypotheses
have some concrete interpretation. This is not so clear and deserves careful
thought. To address the first issue of structured interactions, we could imagine
connecting bayesian learners on different networks to observe network effects on
convergence and innovation rates. To address the second issue, we could imagine
adding communication noise to inter-agent communication, so that agents update
their hypotheses based on imperfect observations. 

The third issue I believe will depend on the application. I think this framework
can be used to impute the ``cultural complexity'' of different fields of study
as revealed by global innovation rates in successful grant applications. 
In that application, we don't really care what the hypotheses are. However, 
the fact that hypotheses are present, though latent, and evolving may be
crucial for appropriately calculating the cultural complexity given innovation
rates. 

As a first step towards these bigger questions, we have demonstrated a
(what kind of) relationship between computational complexity and cultural 
complexity, and innovation rates and cultural complexity. Next I will try 
calculating innovation rates for different NSF divisions from one year to the
next, following the method for analyzing French revolution speeches in 
\cite{Barron2018}. We can see if there exists the same modeled relationship
between innovation rate and cultural complexity, as we could measure through
the optimal number of topics for each directorate in Latent Dirichlet Allocation
\citep{Blei2003,Griffiths2004}.



%\newpage
\bibliographystyle{newapa}
\bibliography{/Users/mt/workspace/papers/library.bib}

\end{document}
